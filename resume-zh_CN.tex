% !TEX TS-program = xelatex
% !TEX encoding = UTF-8 Unicode
% !Mode:: "TeX:UTF-8"

\documentclass{resume}
\usepackage{zh_CN-Adobefonts_external} % Simplified Chinese Support using external fonts (./fonts/zh_CN-Adobe/)
%\usepackage{zh_CN-Adobefonts_internal} % Simplified Chinese Support using system fonts
\usepackage{linespacing_fix} % disable extra space before next section
\usepackage{cite}

\begin{document}
\pagenumbering{gobble} % suppress displaying page number

\name{董显林 | 前端开发实习}

\basicInfo{
  \email{dongxianlin@vadxq.com} \textperiodcentered\ 
  \phone{17679376618} \textperiodcentered\ 
  \github[vadxq]{https://github.com/vadxq}}
 
\section{\faGraduationCap\  教育背景}
\datedsubsection{\textbf{南昌大学}}{2015.09 -- 至今}
\textit{在读本科生}\ 光电信息科学与工程, 预计 2019 年 6 月毕业
%\datedsubsection{\textbf{南昌大学家园工作室}}{2015.10 -- 至今}
%\textit{前端开发}

\section{\faFolderOpen\  工作经历}
\datedsubsection{\textbf{南昌大学家园工作室}}{2015.10 -- 至今}
\textit{前端开发}

\section{\faUsers\ 项目经历}
\datedsubsection{\textbf{南昌大学家园网}}{2018年2月 -- 2018年3月}
% \role{实习}{前端开发}
\begin{itemize}
  \item 基于 Nuxtjs 开发 Vue+SSR+PWA 应用实践
  \item 通过编写第三方插件进行全局错误处理及复用函数提取等
  \item 使用 Docker 进行服务端部署,利用 Certbot 签发 SSL ,使用 Nginx 反向代理
  \item 优化网站的 SEO
\end{itemize}

\datedsubsection{\textbf{云家园信息服务系统}}{2016年7月 -- 至今}
\begin{itemize}
  \item 使用 Vue 完成单页应用开发,目前已有50+二级应用
  \item 提取通用组件,如七牛上传,编辑器等,提高组件复用率,提高开发效率
  \item 封装通用函数及适用业务逻辑函数,如封装 Ajax 处理各种响应
  \item 利用 CDN 实现静态资源缓存与分发,优化不同区域的访问速度,提高用户体验
  \item 曾先后使用 Webhook 和 CI 完成项目自动化持续集成,现通过本地 Build ,推送至 OSS 对象存储
\end{itemize}
\datedsubsection{\textbf{校园卡消费记录}}{2017年7月 -- 2017年8月}
\begin{itemize}
  \item 使用 JavaScript 开发,深度处理杂乱的数据以满足产品需求
  \item 使用 Gulp 按需配置热加载,本地代理和 Less 等文件打包等
 \end{itemize}

%\datedsubsection{\textbf{\LaTeX\ 简历模板}}{2015 年5月 -- 至今}
%\role{\LaTeX, Python}{个人项目}
%\begin{onehalfspacing}
%优雅的 \LaTeX\ 简历模板, https://github.com/billryan/resume
%\begin{itemize}
%  \item 容易定制和扩展
%  \item 完善的 Unicode 字体支持,使用 \XeLaTeX\ 编译
%  \item 支持 FontAwesome 4.5.0
%\end{itemize}
%\end{onehalfspacing}

% Reference Test
%\datedsubsection{\textbf{Paper Title\cite{zaharia2012resilient}}}{May. 2015}
%An xxx optimized for xxx\cite{verma2015large}
%\begin{itemize}
%  \item main contribution
%\end{itemize}

\section{\faCogs\ 技能}
% increase linespacing [parsep=0.5ex]
\begin{itemize}[parsep=0.5ex]
  \item 能够编写语义化的 HTML ,熟悉 CSS 的使用,了解 HTML5/CSS3 ,并能使用部分新特性
  \item 熟悉 JavaScript 语法和基本特性,了解 ES6 新特性,不断实践并在新项目中应用
  \item 了解 Vue/Bootstrap 等常见框架类库的使用
  \item 学习并使用 Gulp , Webpack 等代码构建工具实现前端工程化
  \item 使用 JavaScript + Koa2/fastify 等框架搭建后端服务,能配合 MongoDB 完成 CRUD
  \item 了解 HTTP 常见知识,对 HTTP2/WebSocket 等新知识有所了解
\end{itemize}

%\section{\faHeartO\ 获奖情况}
%\datedline{\textit{第一名}, xxx 比赛}{2013 年6 月}
%\datedline{其他奖项}{2015}

\section{\faInfo\ 其他}
% increase linespacing [parsep=0.5ex]
\begin{itemize}[parsep=0.5ex]
  \item 日常使用 Git 等代码管理与团队协作工具
  \item 日常使用 Linux + VSCode 作为开发环境,熟悉常用命令与配置
  \item 日常使用 Docker 完成项目的构建与部署,完成 Nginx 的配置,实现反向代理等功能
\end{itemize}

%% Reference
%\newpage
%\bibliographystyle{IEEETran}
%\bibliography{mycite}
\end{document}
